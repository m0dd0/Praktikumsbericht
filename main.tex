%% bare_conf.tex
%% V1.4b
%% 2015/08/26
%% by Michael Shell
%% http://www.ieee.org/

\documentclass[conference]{IEEEtran}
\usepackage{cite}
% \cite{} output to follow that of the IEEE. Loading the cite package will
% result in citation numbers being automatically sorted and properly
% "compressed/ranged". e.g., [1], [9], [2], [7], [5], [6] without using
% cite.sty will become [1], [2], [5]--[7], [9] using cite.sty. cite.sty's
% \cite will automatically add leading space, if needed. Use cite.sty's
% noadjust option (cite.sty V3.8 and later) if you want to turn this off
% such as if a citation ever needs to be enclosed in parenthesis.

\usepackage{amsmath}
\usepackage{amsfonts}
% A popular package from the American Mathematical Society that provides
% many useful and powerful commands for dealing with mathematics.
% Note that the amsmath package sets \interdisplaylinepenalty to 10000
% thus preventing page breaks from occurring within multiline equations. Use:
%\interdisplaylinepenalty=2500

\usepackage{algorithmic}
% algorithmic.sty was written by Peter Williams and Rogerio Brito.
% This package provides an algorithmic environment fo describing algorithms.
% You can use the algorithmic environment in-text or within a figure
% environment to provide for a floating algorithm. Do NOT use the algorithm
% floating environment provided by algorithm.sty (by the same authors) or
% algorithm2e.sty (by Christophe Fiorio) as the IEEE does not use dedicated
% algorithm float types and packages that provide these will not provide
% correct IEEE style captions. 

\usepackage{array}
% Frank Mittelbach's and David Carlisle's array.sty patches and improves
% the standard LaTeX2e array and tabular environments to provide better
% appearance and additional user controls. As the default LaTeX2e table
% generation code is lacking to the point of almost being broken with
% respect to the quality of the end results, all users are strongly
% advised to use an enhanced (at the very least that provided by array.sty)
% set of table tools. a
% http://www.ctan.org/pkg/array

\usepackage[caption=false,font=footnotesize]{subfig}
% subfig.sty, written by Steven Douglas Cochran, is the modern replacement
% for subfigure.sty, the latter of which is no longer maintained and is
% incompatible with some LaTeX packages including fixltx2e. However,
% subfig.sty requires and automatically loads Axel Sommerfeldt's caption.sty
% which will override IEEEtran.cls' handling of captions and this will result
% in non-IEEE style figure/table captions. To prevent this problem, be sure
% and invoke subfig.sty's "caption=false" package option (available since
% subfig.sty version 1.3, 2005/06/28) as this is will preserve IEEEtran.cls
% handling of captions.
% http://www.ctan.org/pkg/subfig

%\usepackage{stfloats}
% stfloats.sty was written by Sigitas Tolusis. This package gives LaTeX2e
% the ability to do double column floats at the bottom of the page as well
% as the top. (e.g., "\begin{figure*}[!b]" is not normally possible in
% LaTeX2e). It also provides a command:
%\fnbelowfloat
% to enable the placement of footnotes below bottom floats (the standard
% LaTeX2e kernel puts them above bottom floats). This is an invasive package
% which rewrites many portions of the LaTeX2e float routines. It may not work
% with other packages that modify the LaTeX2e float routines. The latest
% version and documentation can be obtained at:
% http://www.ctan.org/pkg/stfloats
% Do not use the stfloats baselinefloat ability as the IEEE does not allow
% \baselineskip to stretch. Authors submitting work to the IEEE should note
% that the IEEE rarely uses double column equations and that authors should try
% to avoid such use. Do not be tempted to use the cuted.sty or midfloat.sty
% packages (also by Sigitas Tolusis) as the IEEE does not format its papers in
% such ways.
% Do not attempt to use stfloats with fixltx2e as they are incompatible.

%\usepackage{url}
% url.sty was written by Donald Arseneau. It provides better support for
% handling and breaking URLs. url.sty is already installed on most LaTeX
% systems. The latest version and documentation can be obtained at:
% http://www.ctan.org/pkg/url
% Basically, \url{my_url_here}.

% *** Do not adjust lengths that control margins, column widths, etc. ***
% *** Do not use packages that alter fonts (such as pslatex).         ***
% There should be no need to do such things with IEEEtran.cls V1.6 and later.
% (Unless specifically asked to do so by the journal or conference you plan
% to submit to, of course. )

% correct bad hyphenation here
\hyphenation{op-tical net-works semi-conduc-tor}

\begin{document}
% paper title
% Titles are generally capitalized except for words such as a, an, and, as,
% at, but, by, for, in, nor, of, on, or, the, to and up, which are usually
% not capitalized unless they are the first or last word of the title.
% Linebreaks \\ can be used within to get better formatting as desired.
% Do not put math or special symbols in the title.
\title{Bare Demo of IEEEtran.cls\\ for IEEE Conferences}


% author names and affiliations
% use a multiple column layout for up to three different
% affiliations
\author{\IEEEauthorblockN{Moritz Hesche}
    \IEEEauthorblockA{
        Karlsruhe Institute of Technology \\
        Karlsruhe, Germany \\
        Email: mo.hesche@gmail.com}
}
% \and
% \IEEEauthorblockN{Homer Simpson}
% \IEEEauthorblockA{Twentieth Century Fox\\
% Springfield, USA\\
% Email: homer@thesimpsons.com}
% \and
% \IEEEauthorblockN{James Kirk\\ and Montgomery Scott}
% \IEEEauthorblockA{Starfleet Academy\\
% San Francisco, California 96678--2391\\
% Telephone: (800) 555--1212\\
% Fax: (888) 555--1212}}

% conference papers do not typically use \thanks and this command
% is locked out in conference mode. If really needed, such as for
% the acknowledgment of grants, issue a \IEEEoverridecommandlockouts
% after \documentclass

% make the title area
\maketitle

% As a general rule, do not put math, special symbols or citations
% in the abstract
\begin{abstract}
    In the published literature there is no lack of algorithms and concepts for robotic object grasping.
Most of the proposed algorithm seem to perform well in the respective environements they were evaluated in.%environments with grasp success rate ov.
However different a priori assumptions made in the problem formulation and varying evaluation environments make it challenging to compare the results of
the different algorithms in a meaningful way.
% Often no or less effort is made to put the algorithms in different setting and analyse their performance on.
Especially with deep learning based algorithms it is hard to estimate the meaningfullness of presented results as the underlying networks are often trained on a specific dataset
which might causes the algorithms to generalize bad to other environement. % and are not easily transferable to other datasets.
In this report we present a benchmark to compare the performance of different grasping algorithms in a common simulated environement using the YCB object dataset.
Using a shared task, environment and objects to grasp we can directly compare the differences in performance between the different approaches to robotic grasping.
% -- done
\end{abstract}

% For peerreview papers, this IEEEtran command inserts a page break and
% creates the second title. It will be ignored for other modes.
\IEEEpeerreviewmaketitle

\section{Introduction}
Introduction here balasdasd
\cite{sundermeyer2021contact}
\cite{kumra2020antipodal}

\subsection{subsection}

\subsubsection{subsubsection}

% \hfill mds
% \hfill August 26, 2015

\section{Benchmark Methology}
\subsection{Software Framework}
\subsection{Metrics}

\section{Benchmarked Algorithms}
\subsection{GrConvNet}
\subsection{ContactGraspNet}
\subsection{GGCNN}
\subsection{Comparison}

\section{Limitations and Differences}
\subsection{GrConvNet}
\subsection{ContactGraspNet}
\subsection{GGCNN}

\section{Necessary Adaptions}
\subsection{GrConvNet}
\subsection{ContactGraspNet}
\subsection{GGCNN}

\section{Results}
\subsection{GrConvNet}
\subsection{ContactGraspNet}
\subsection{GGCNN}

\section{Error Analysis}
\subsection{GrConvNet}
\subsection{ContactGraspNet}
\subsection{GGCNN}

\section{Conclusion}
% \input{0?_conclusion}

% conference papers do not normally have an appendix

\section*{Acknowledgment}
% \input{0?_acknowledgments}

% trigger a \newpage just before the given reference
% number - used to balance the columns on the last page
% adjust value as needed - may need to be readjusted if
% the document is modified later
%\IEEEtriggeratref{8}
% The "triggered" command can be changed if desired:
%\IEEEtriggercmd{\enlargethispage{-5in}}

\bibliographystyle{IEEEtran}
\bibliography{99_bibliography/bibliography}

\end{document}
