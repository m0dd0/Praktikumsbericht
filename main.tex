%% bare_conf.tex
%% V1.4b
%% 2015/08/26
%% by Michael Shell
%% http://www.ieee.org/

\documentclass[conference]{IEEEtran}
\usepackage{cite}
% \cite{} output to follow that of the IEEE. Loading the cite package will
% result in citation numbers being automatically sorted and properly
% "compressed/ranged". e.g., [1], [9], [2], [7], [5], [6] without using
% cite.sty will become [1], [2], [5]--[7], [9] using cite.sty. cite.sty's
% \cite will automatically add leading space, if needed. Use cite.sty's
% noadjust option (cite.sty V3.8 and later) if you want to turn this off
% such as if a citation ever needs to be enclosed in parenthesis.

\usepackage{amsmath}
% A popular package from the American Mathematical Society that provides
% many useful and powerful commands for dealing with mathematics.
% Note that the amsmath package sets \interdisplaylinepenalty to 10000
% thus preventing page breaks from occurring within multiline equations. Use:
%\interdisplaylinepenalty=2500

\usepackage{algorithmic}
% algorithmic.sty was written by Peter Williams and Rogerio Brito.
% This package provides an algorithmic environment fo describing algorithms.
% You can use the algorithmic environment in-text or within a figure
% environment to provide for a floating algorithm. Do NOT use the algorithm
% floating environment provided by algorithm.sty (by the same authors) or
% algorithm2e.sty (by Christophe Fiorio) as the IEEE does not use dedicated
% algorithm float types and packages that provide these will not provide
% correct IEEE style captions. 

\usepackage{array}
% Frank Mittelbach's and David Carlisle's array.sty patches and improves
% the standard LaTeX2e array and tabular environments to provide better
% appearance and additional user controls. As the default LaTeX2e table
% generation code is lacking to the point of almost being broken with
% respect to the quality of the end results, all users are strongly
% advised to use an enhanced (at the very least that provided by array.sty)
% set of table tools. a
% http://www.ctan.org/pkg/array

\usepackage[caption=false,font=footnotesize]{subfig}
% subfig.sty, written by Steven Douglas Cochran, is the modern replacement
% for subfigure.sty, the latter of which is no longer maintained and is
% incompatible with some LaTeX packages including fixltx2e. However,
% subfig.sty requires and automatically loads Axel Sommerfeldt's caption.sty
% which will override IEEEtran.cls' handling of captions and this will result
% in non-IEEE style figure/table captions. To prevent this problem, be sure
% and invoke subfig.sty's "caption=false" package option (available since
% subfig.sty version 1.3, 2005/06/28) as this is will preserve IEEEtran.cls
% handling of captions.
% http://www.ctan.org/pkg/subfig

%\usepackage{stfloats}
% stfloats.sty was written by Sigitas Tolusis. This package gives LaTeX2e
% the ability to do double column floats at the bottom of the page as well
% as the top. (e.g., "\begin{figure*}[!b]" is not normally possible in
% LaTeX2e). It also provides a command:
%\fnbelowfloat
% to enable the placement of footnotes below bottom floats (the standard
% LaTeX2e kernel puts them above bottom floats). This is an invasive package
% which rewrites many portions of the LaTeX2e float routines. It may not work
% with other packages that modify the LaTeX2e float routines. The latest
% version and documentation can be obtained at:
% http://www.ctan.org/pkg/stfloats
% Do not use the stfloats baselinefloat ability as the IEEE does not allow
% \baselineskip to stretch. Authors submitting work to the IEEE should note
% that the IEEE rarely uses double column equations and that authors should try
% to avoid such use. Do not be tempted to use the cuted.sty or midfloat.sty
% packages (also by Sigitas Tolusis) as the IEEE does not format its papers in
% such ways.
% Do not attempt to use stfloats with fixltx2e as they are incompatible.

%\usepackage{url}
% url.sty was written by Donald Arseneau. It provides better support for
% handling and breaking URLs. url.sty is already installed on most LaTeX
% systems. The latest version and documentation can be obtained at:
% http://www.ctan.org/pkg/url
% Basically, \url{my_url_here}.

% *** Do not adjust lengths that control margins, column widths, etc. ***
% *** Do not use packages that alter fonts (such as pslatex).         ***
% There should be no need to do such things with IEEEtran.cls V1.6 and later.
% (Unless specifically asked to do so by the journal or conference you plan
% to submit to, of course. )

% correct bad hyphenation here
\hyphenation{op-tical net-works semi-conduc-tor}

\begin{document}
% paper title
% Titles are generally capitalized except for words such as a, an, and, as,
% at, but, by, for, in, nor, of, on, or, the, to and up, which are usually
% not capitalized unless they are the first or last word of the title.
% Linebreaks \\ can be used within to get better formatting as desired.
% Do not put math or special symbols in the title.
\title{Bare Demo of IEEEtran.cls\\ for IEEE Conferences}


% author names and affiliations
% use a multiple column layout for up to three different
% affiliations
\author{\IEEEauthorblockN{Moritz Hesche}
    \IEEEauthorblockA{
        Karlsruhe Institute of Technology \\
        Karlsruhe, Germany \\
        Email: mo.hesche@gmail.com}
}
% \and
% \IEEEauthorblockN{Homer Simpson}
% \IEEEauthorblockA{Twentieth Century Fox\\
% Springfield, USA\\
% Email: homer@thesimpsons.com}
% \and
% \IEEEauthorblockN{James Kirk\\ and Montgomery Scott}
% \IEEEauthorblockA{Starfleet Academy\\
% San Francisco, California 96678--2391\\
% Telephone: (800) 555--1212\\
% Fax: (888) 555--1212}}

% conference papers do not typically use \thanks and this command
% is locked out in conference mode. If really needed, such as for
% the acknowledgment of grants, issue a \IEEEoverridecommandlockouts
% after \documentclass

% make the title area
\maketitle

% As a general rule, do not put math, special symbols or citations
% in the abstract
\begin{abstract}
    In the published literature ther is no lack of algorithms and concepts for object grasping.
Most of the proposed seem to perform well in the tested environments.
However different assumptions made it challenging to compare the results of the different algorithms.


These difference include but are not limited to

\end{abstract}

% For peerreview papers, this IEEEtran command inserts a page break and
% creates the second title. It will be ignored for other modes.
\IEEEpeerreviewmaketitle

\section{Introduction}
% different metrics used in papers
% different robots, different datasets
% different challenges (e.g. cluttered environment)
% different grasp representations
% different environments
% different evaluation metrics

% \subsection{subsection}

% \subsubsection{subsubsection}


\section{Benchmark Methology}
In this section we describe the methodology used to evaluate the performance of the compared grasp algorithms.
First we focus on the technical aspects of the software framework used to evaluate the algorithms.
This includes a framework to handle the different grasp representations returned by the different algorithms.
The second section describes how we quntify the performance of a algorithm is measured in the used benchmark setting and how the results should be interpreted.
Lastly a short description of the used objects is given.

\subsection{Software Framework Components}
% Motivation for the framework: different grasp representations, different input data
As the different algorithm all use different grasp representations and require different different input data, we integrated the them into
a software framework which offers a clean interfact between the executing environment and the grasp algorithm itself.
Another element of the used benchmarking suite is a dedicated benchmark runner. This allows for the execution of consistent and reproducable benchmark runs.

% Desccription of the ROS Service interface: input, output, wrapper, integration with common representation
The interface between executing environement and executing environement is implemented as a ROS service which can be called from an arbitrary environment.
An clearly defined set of observation can be passed to the service call and can be used by the algorithm to generate a grasp.
The observation currently include the following data:
\begin{itemize}
    \item RGB image of the scene
    \item Depth image of the scene
    \item Pointcloud of the scene
    \item Camera pose
    \item Camera intrinsics
    \item Number of grasps to be generated
\end{itemize}
After the execcution of an grasping algorithm the service returns unified grasp representation which is used to execute the grasp in a simulated or real environment.
Independent of the grasp representation used by the algorithm, the service always returns a list of grasps where a grasp is defined by an 6D-Pose, a gripper width and
an algorithm agnostic estimated performance score of the grasp.

% ROS service as wrapper
For each benchmarked algorithm a ROS service was implemented which are all contained within a single repository to allow for easy integration of new algorithms.
Each service acts as a wrapper around the original implementaion of the algorithms.
This especially includes proper conversion of the described observation data into the input format required by the algorithm as well as the conversion of
the algorithm specific grasp representation to the unified grasp representation.
In section \ref{sec:benchmarking} we will discuss the necessary conversion and adaptions of the data in more detail.

% Simulation environment: description, limitations
A main challenge in benchmarking different grasp algorithms is to find a setting which allows to offer an challenging environment for the algorithms while
still being able to evaluate the algorithms in a reasonable amount of time.
Our benchmark is therefore based on a simulation environment which is based on the mujoco physics engine \cite{}.
However, there is no reason why the described benchmarking framework is limited to a simulation environment.
The simulation environment consists of a table where the objects are placed on and a robot arm and is displayed in Figure~\ref{fig:env}.


\subsection{Used Objects}
% YCB dataset: desciption, why used
For benchmarking the different algorithms we used the used the YCB dataset \cite{}.
This dataset contains ?? different household objects of different sizes and shapes and is commonly used for benchmarking grasp algorithms.
This covers fully convex objects like cubes or spheres but also objects with complex geometries.
While the most deep learning based grasping algorithms are trained on a much larger varieties of objects, the YCB dataset is a good starting point for
benchmarking grasp algorithms with a reasonable computational cost.
The variety of the dataset object still allows for a representative result.

% restriction on the objects
A key limitation in robotic grasping are the physical limitations of the robot gripper.
As an a priori assumption we therefore limited the evaluation of the algorithms to objects which can be fitted between the fingers of the gripper.
This is a reasonable assumption as the gripper is the only part of the robot which is in direct contact with the object.
% For all the remaining objects in our test dataset we attempted 50 grasps for each object and algorithm.


\subsection{Metrics}
In each attempt we randomly placed the object on a flat table and randomly rotated the object.
However, we did not allow the object to be placed on the table in a way that the object is not fully visible to the camera or be outside the range of the robot.
After setting up the scene the grasp is executed.

% Grasping procedure
The execution of a grasp consists of a sequence of steps.
First the robot is moved to constant home position. This position is selected in away that the robot is not covering any object geometries from the camera viewpoint.
After the robot reached it's home position the observation data is collected and the ROS Service is called.
After receiving the grasp proposals the robot is moved to a hover position and oriented as defined by the received grasp message.
The hover position corresponds to the actual grasp position but translated along the grasp axis by a fixed distance.
This approach allows to reduce errors occuring from illegal approaching to the grasp object.
The actual grasp is then executed by opening the gripper, moving linearly to the target position, closing the position and moving back to the hover position.
In last step of the executed sequence the gripper is moved back to the home position.

The success of a grasp is defined as the gripper being able to move the object to a predefined position so that the center of the object is no more than
?? away from the center of the gripper.

We repeat this procedure for each object in the dataset and for each algorithm 50 times.
Using multiple

\section{Benchmarked Algorithms}
We selected 3 different algorithms for benchmareking.
While the benchmark focusses on deep learning based grasping algorithm the algorithms were selected so that different approaches to the task of object grasping
are represented.
Two of the benchmarked algorithms use image based inputs and return planar grasps.
The third algorithm uses pointcloud based inputs and returns 6D grasps.
This selection might allow to get an deeper understanding of the strengths and weaknesses of the different approaches and input modalities used.
A more practical aspect in the selection of the benchmarked algorithms was the availability of an publicly available source code.

% TODO reference
To allow for a better understanding of the results which we will discuss in \ref{} we will give a short description over the benchmarked
algorithms in the following subsections focussing on the algorithmic approaches.
Results and error analysis will be discussed in \ref{}.
% -- done


\subsection{GrConvNet}
% general
The GrConvNet algorithm described in \cite{kumra2020antipodal} processes an 4-channel RGBD image of the scene with a deep convolutional network to predict multiple
planar grasp candidates.
% network architecture
The network architecture consists of a set of convolutional layers followed by a symmetrical set of transposed convolutional layers.
To achieve better performance the used network architecture utilizes batch normalization layers, ReLU activation functions and residual connections.
% input and output
Due to the symmetrical architecture the first dimensions of the network output correspond to the dimensions of the input image.
As a consequence the output can be interpreted as a distribution of grasp candidates in the image space.
Each grasp in the image space is represented by it's pixel coordinate, a grasp width $W$, a grasp angle $\Theta_i$ referencing the image u-axis and a grasp quality value $Q$:
$G_i = (u,v,\Theta_i, W_i, Q)$
% postprocessing
To obtain the elite grasp candidates from the grasp distribution non-maximum suppression algorithm is used to filter grasp candidates which are below a predefined
quality value or are too close to another but better grasp candidate.
In an upfollowing postprocessing stage the grasp representation of the elite grasp candidates in the image space are transformed to the world space using the
cameras extrinsic and intrinsic parameters.
% TODO: maybe visulaization here
% training
% TODO cite
The training of the network was done disjunct on the Cornell \cite{}, Dexnet \cite{} and Jacquard \cite{} dataset. To the best of our knowledge no training was executed on a combination of the datasets.
As an training objective the sum of the log-likelihood of the elite grasp candidates being close to the ground gruth grasp candidates was used.
For a training dataset with $n$ images and $m$ grasp candidates this objective can be expressed as:
$L = \sum_{i=1}^n \sum_{j=1}^m \log p(G_j|I_i)$
where $p(G_j|I_i)$ is the probability of the $j$th grasp candidate in the $i$th image being similar enough to the ground truth grasp candidate.
The simmilarity of a predicted grasp to the ground truth grasp is determined by the intersection over union of the grap rectangles and the differenc in their grasp angles.
Due to empirical analysis of the performance Huber Loss was selected as loss function.
% -- done


\subsection{ContactGraspNet}
% general
In \cite{sundermeyer2021contact} the point cloud based \textit{ContactGraspNet} algorithm is presented which can be viewed as a successor of the method described in \cite{}.
The algorithm is based on the idea that in a partially observable environment, the search space for grasp candidates can be limited to the set of grasps which are
in contact with one of the points in the observed pointcloud.
% network architecture
%  TODO
% input and output
In contrast to \texit{GrConvNet} and \texit{GGCNN} the \textit{ContactGraspNet} is not limited to grasps parallel to a pedefined surface.
Instead the algorithm can be used to gather grasps with an arbitrary orientation.
Consequently a grasp is represented by a 6D pose consisting of a position and an orientation:
$$ G = (R_g, t_g) = $$
Each observed point can be used as an anchor point for a grasp.
A visulization of this grasp representation in our simulation environment is shown in \ref{}.
The network architecture described uses the full pointcloud to obtain a number of grasp candidates for all objects present in the pointcloud.
However, the presented implementation in \ref{} also adds the ability to filter the grasp candidate based on a provided segmentation.


% training
Due to it's pointcloud based anatomy the underlying network wass
Instead they utilized ACRONYM dataset and the Shapenet dataset
In contrast
The grasp axis can be oriented in arbitrary direction


\subsection{GGCNN}
Siimilarly to the \textit{GrConvNet} algorithm the \textit{GGCNN} algorithm based on image inputs and returns image based grasp representations.
\subsection{Comparison}


% \section{Limitations and Differences}
% % \subsection{GrConvNet}
% \subsection{ContactGraspNet}
% \subsection{GGCNN}

In the previous section we described the different algorithms which were benchmarked.
The different approaches used by these algorithms lead to different strengths and weaknesses.
For a better undersanding of the results we will now discuss the limitations of the different algorithms and the differences between them.

The planar grasp algorith have the advantage that they are fast and can be used in real-time.
A major downside however is that the best, or in some cases the onyl possible grasp axis are not parallel to the image space.
Also it becomes increasingly challengin to obtain a suitable transformation for camera orientation which are not parallel to the surface the object is lying on.

The issues arising from the reduced orientation space of planar grasps are bypassed by approaches which use fully spatial grasp representations.
The unrestricted orientation space however also leads to a challenge in approaching the object without colliding with the object.
The analysed ContactGraspNet algorithm proposes no direct way of handling this issue.
However, as argued in \cite{} the algorithm used only points within the view area of the camera and therfore reducing the chance og generating grasp poses which will
lead to a collision whilc approachin the grasp position.





\section{Necessary Adaptions}
% Introductions, scopr of adaptions
While all of the bencharked algorithm claimed to achieve a success rate of over 90\% in their respective evaluation settings, a number of adaptions were necessary to
make them work in our simulation environment at all and reduce the influences caused by the different setting.
In general we tried to adapt the algorithm itself as less as possible to keep the results as comparable as possible.
We never retrained the underlying networks on our data or made any othes algorithmic changed to the bencharked algorithms.
However, we had to make some adaptions to the pre- and postprocessing steps of the algorithms to make them work with the benchmarking framework described in \ref{}.
The specific adaptions made to each algorithm are described in the following subsections.

% Packaging
Another asect of the adaptions was the packaging of the benchmarked algorithms.
None of the benchmarked algorithms were implemented to work as standalone applications.
To use the implementation from within a running ROS serivce we had to refactor the code in order to package it as a standalone application.
Also the dependency management of the benchmarked algorithms needed to be adapted.
% -- done


\subsection{GrConvNet}
% general

% Center cropping
The preprocessing stage of the GrConvNet implementaion incorporrates a simple center cropping of the input image to the desired input size of $224x224$ pixels.
This works flawless as long as the object of interest is centered in the image and covers only an area smaller than the cropped image area.
However, when using simulation data with different camera perspectives, the object of interest is not always centered in the image.
As a workaround we replaced the center cropping mechanism with a combination of square cropping and resizing.
This approach allows us to obtain the correct input dimension while still keeping the object of interest in the center of the image and also in reasonable size.
% TODO (refer in) problem due to overfitteing

% segmentation
We observed that an additional previous segmentation is essential for the GrConvNet algorithm to work properly in pur environement.
We assume that this comes due to the fact that the dataset the network was trained on a dataset containing white background images only and no data augmentation
was done to counteract for this imbalance.
Having a multicolored background with different color gradients in the backgroud therfore leads to the network predicting grasps in the background and not on the target object.
Addin an object segmentation layer to the preprocessing of the images resulted in a significant improvement of the grasp quality.
% which parameter set was used

% conversions to world space
The used conversion in the referenced implementation were only valid for the edge case in which we can assume that the camera is parallel to the ground plane.
To create a more general solution we first transformed the used image based grasp representation explained in section \ref{} to a more general representation
using a end effector rotation matrix $R_g^I$ in image space instead of a single angle $\alpha$.
Since the grasps are restricted to be parallel to the ground plane, we can write the orientation and center point in image space as follows:
$$ R_g^I =  \begin{bmatrix}
        - \sin\alpha & \cos\alpha & 0  \\
        \cos\alpha   & \sin\alpha & 0  \\
        0            & 0          & -1
    \end{bmatrix}
$$
This transformation also incorporates the fact that in our simulated environmend the grasp axis is defined along the y-axis of the end effector coordinate system.
In the \cite{kumra2020antipodal} paper and also in the accompanying implementation no information were provide on how any information on how the grasp height was estimated.
We made therefore the additional assumption that the height of the grasp can estimated as the height at the predicted grasp point offsetted by a constant, gripper specific value.
In consequence the grasp center point in camera coordinates can be calculated as follows from the grasp center coordinates $u$ and $v$ in camera space and the estimated grasp height $p_gz^C$:
$$ t_g^C = \begin{pmatrix}
        (u - cx) * p_cam_z / fx
        p_cam_y = (p_img_y - cy) * p_cam_z / fy \\
        (p_img_x - cx) * p_cam_z / fx
        p_cam_y = (p_img_y - cy) * p_cam_z / fy \\
        p_gz^C
    \end{pmatrix} $$ % TODO

As a last step we can use the following formulas to transform the grasp representation from camera space to world space:
% TODO: add formulas


\subsection{ContactGraspNet}
% number of points as input
During development we observed significant difference in the grasp quality depending on the nuber of points being contained in the used pointcloud.
This can be explained by the preprocessing stage of the ContactGraspNet algorithm.
To reduce the computational complexity of the Pointnet++ inference, the pointcloud is randomly downsampled to 20000 points.
For pointclouds which are in the order of magnitude of 100000 points, this can results in a significant loss of information as we can not assume anymore that all parts of
the original pointlcloud are covered by the sampled subset of points.
The pointcloug generation in our simulation environment is based on depth image. For each pixel in the depth image a point is generated in the pointcloud.
We can exploit this mechanism to generate a pointcloud with a fixed number of points by simply using a lower resolution of the underlying depth image.

% different segmentation
The ContactGraspNet algorithm already implements a method for object segmentation.
This segmentation is uses 3D bounding boxes generated from a 2D segmentation image in order to filter the pointcloud directly.
In our case an additional previous segmentation was beneficial to the grasp quality.
We therfore reduced the pointlcoud to the area around the object of interest keeping the local context of the object but suppressing background elements like floor
or the robot arm.
We suspect that this was caused by a number of grasps being generated for geometries that are not part of the object of interest.
As a result less or in some cases no grasps were generated for the object of interest if too much auxiliary geometries were present in the scene.

% coordinate transformations
While directly offering a 6D grasp representation, there was still a conversion of the grasp pose necessary in order to obtain a correct grasp in the simulation environement.
This arises from different end effector coordinate systems used by the simulation and the algorithm.
Both representation define the z-axis as the axis along which the object is attempted in order to be grasped.
However, the simulation uses the y-axis as the axis along which the object is grasped while the algorithm uses the x-axis.
% This results in the following conversion of the grasp orientation matrix beeing necessary:
The different resulting rotation matrices can be calculated as follows:
$$ R_g^paper =  \begin{bmatrix}
        | & |          & | \\
        b & a \times b & a \\
        | & |          & |
    \end{bmatrix}
$$
$$
    \begin{bmatrix}
        |           & | & | \\
        -a \times b & b & a \\
        |           & | & |
    \end{bmatrix}
$$

The change of the sign of $a \times b$ is necessary to keep the resulting end effector coordinte system orthogona and clockwise.
% -- done

\subsection{GGCNN}


\section{Results}
A full overview of the grasp success rate of different algorithms on all tested objects can be found in the apendix.

\subsection{GrConvNet}
\subsection{ContactGraspNet}
\subsection{GGCNN}
% results over oject sizes

% results over object types (convec concav)

\section{Error Analysis}
\subsection{GrConvNet}
% grasp height
As described in \ref{} we had to use a coarse approximation of the grasp height in order to generate grasp candidates for the \textit{GrConvNet} algorithm.
This approximation however is error prone to object with a non-planar surface.

% overfitting
Abother indicator that \textit{GrConvNet} might suffer from an overfitting problem is demonstrated in \ref{}.
For a plain

% small gripper width
A factor
We observed many cases were the grasp axis seemed to be generated with a proper orientation but the grasp center point was not placed correctly.
Especialy in cases where there is not much tolerance for the grasp center point due to the object size along the grasp axis limited maximum gripper.
It must be noted that this is an robot specific issue and might not occur for end effectors which offer a larger maximum gripper width.
However, in cluttered environements this can lead to a significant decrease in performance as a offseted grasp center point can lead to a collision
with other objects in the scene.

\subsection{ContactGraspNet}
In some cases the \textit{ContactGraspNet} algorithm wasn't able to generate any grasp candidates for a given object.
We assume that whic was caused by filtering executed in the \textit{ContactGraspNet} algorithm.
This means that underlying network was able to generate grasp candidates but they were filtered due to the applied constraints like collision avoidance
and object segmentation.


\subsection{GGCNN}
Similar to the \textit{GrConvNet} the \textit{GGCNN} algorithm also had problems in estimating the grasp height correctly.
In many of the observed cases the generated grasps

\subsection{Comparison}


In the previous section we described the different algorithms which were benchmarked.
The different approaches used by these algorithms lead to different strengths and weaknesses.
For a better undersanding of the results we will now discuss the limitations of the different algorithms and the differences between them.

The planar grasp algorith have the advantage that they are fast and can be used in real-time.
A major downside however is that the best, or in some cases the onyl possible grasp axis are not parallel to the image space.
Also it becomes increasingly challengin to obtain a suitable transformation for camera orientation which are not parallel to the surface the object is lying on.

The issues arising from the reduced orientation space of planar grasps are bypassed by approaches which use fully spatial grasp representations.
The unrestricted orientation space however also leads to a challenge in approaching the object without colliding with the object.
The analysed ContactGraspNet algorithm proposes no direct way of handling this issue.
However, as argued in \cite{} the algorithm used only points within the view area of the camera and therfore reducing the chance og generating grasp poses which will
lead to a collision whilc approachin the grasp position.





\section{Conclusion}
% \input{0?_conclusion}

% conference papers do not normally have an appendix

\section*{Acknowledgment}
% \input{0?_acknowledgments}

% trigger a \newpage just before the given reference
% number - used to balance the columns on the last page
% adjust value as needed - may need to be readjusted if
% the document is modified later
%\IEEEtriggeratref{8}
% The "triggered" command can be changed if desired:
%\IEEEtriggercmd{\enlargethispage{-5in}}

\bibliographystyle{IEEEtran}
\bibliography{99_bibliography/bibliography}

\end{document}
