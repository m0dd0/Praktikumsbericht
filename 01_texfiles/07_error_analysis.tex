\subsection{GrConvNet}
% grasp height
As described in \ref{} we had to use a coarse approximation of the grasp height in order to generate grasp candidates for the \textit{GrConvNet} algorithm.
This approximation however is error prone to object with a non-planar surface.

% overfitting
Abother indicator that \textit{GrConvNet} might suffer from an overfitting problem is demonstrated in \ref{}.
For a plain

% small gripper width
A factor
We observed many cases were the grasp axis seemed to be generated with a proper orientation but the grasp center point was not placed correctly.
Especialy in cases where there is not much tolerance for the grasp center point due to the object size along the grasp axis limited maximum gripper.
It must be noted that this is an robot specific issue and might not occur for end effectors which offer a larger maximum gripper width.
However, in cluttered environements this can lead to a significant decrease in performance as a offseted grasp center point can lead to a collision
with other objects in the scene.

\subsection{ContactGraspNet}
In some cases the \textit{ContactGraspNet} algorithm wasn't able to generate any grasp candidates for a given object.
We assume that whic was caused by filtering executed in the \textit{ContactGraspNet} algorithm.
This means that underlying network was able to generate grasp candidates but they were filtered due to the applied constraints like collision avoidance
and object segmentation.


\subsection{GGCNN}
Similar to the \textit{GrConvNet} the \textit{GGCNN} algorithm also had problems in estimating the grasp height correctly.
In many of the observed cases the generated grasps