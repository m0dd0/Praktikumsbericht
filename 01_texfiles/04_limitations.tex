% \subsection{GrConvNet}
% \subsection{ContactGraspNet}
% \subsection{GGCNN}

In the previous section we described the different algorithms which were benchmarked.
The different approaches used by these algorithms lead to different strengths and weaknesses.
For a better undersanding of the results we will now discuss the limitations of the different algorithms and the differences between them.

The planar grasp algorith have the advantage that they are fast and can be used in real-time.
A major downside however is that the best, or in some cases the onyl possible grasp axis are not parallel to the image space.
Also it becomes increasingly challengin to obtain a suitable transformation for camera orientation which are not parallel to the surface the object is lying on.

The issues arising from the reduced orientation space of planar grasps are bypassed by approaches which use fully spatial grasp representations.
The unrestricted orientation space however also leads to a challenge in approaching the object without colliding with the object.
The analysed ContactGraspNet algorithm proposes no direct way of handling this issue.
However, as argued in \cite{} the algorithm used only points within the view area of the camera and therfore reducing the chance og generating grasp poses which will
lead to a collision whilc approachin the grasp position.



